\documentclass[main.tex]{subfiles}
\begin{document}
\subsection{Calculate $A(6)$ and $A(10)$ in $\mathfrak{su}(3)$}
We use the following relations
\begin{align}
A(\overline{D})=&-A(D)\\
A(D_1\oplus D_2)=&A(D_1)+A(D_2)\\
A(D_1\otimes D_2)=&\dim{(D_1)}A(D_2)+\dim{(D_2)}A(D_1)
\end{align}
and 
\begin{align}
3\otimes3&=6\oplus\overline{3}\\
6\otimes3&=10\oplus8\\
3\otimes\overline{3}&=8\oplus1.
\end{align}

\begin{equation}
\Tr{\left[\{T_a^3,T_b^3\}T_c^3\right]}\equiv A(3)d_{abc}
\end{equation}
So that $A(3)=1$. Then,
\begin{align}
A(6)=&A(3\otimes3)-A(\overline{3})\\
=&A(3\otimes3)+A(3)\\
=&3A(3)+3A(3)+A(3)=7\\
\end{align}

For the $10$,
\begin{align}
A(10)=&A(6\otimes3)-A(8)\\
=&6A(3)+3A(6)-A(8)\\
=&6+21-A(8)
\end{align}
but the $8$ is anomaly-free ($A(8)=0$) because $8=\overline{8}$ so
\begin{equation}
A(8)=A(\overline{8})=-A(8)\implies A(8)=0.
\end{equation}
So
\begin{equation}
A(10)=27.
\end{equation}
\subsection{Show that the $A(10)_{\mathfrak{su}(5)}=A(5)_{\mathfrak{su}(5)}$ in $\mathfrak{su}(5)$}
The anomaly of the fundamental representation of $\mathfrak{su}(N)$ may be calculated by calculating the $\mathfrak{su}(3)$ subalgebra of $\mathfrak{su}(N)$ under which the $N=[1]$
 transforms like a single $3$ and $N-3$ singlets. 
 
Under the $\mathfrak{su}(3)$ subalgebra the $5=[1]$ of $\mathfrak{su}(5)$
\begin{equation}
\young(\hfil)\rightarrow\young(\hfil)\oplus\bullet\oplus\bullet=3\oplus1\oplus1
\end{equation}  
which gives the anomaly $A(5)_{\mathfrak{su}(5)}=A(3)+A(1)+A(1)=1+0+0=1$.

Under the $\mathfrak{su}(3)$ subalgebra the $10=[2]$ of $\mathfrak{su}(5)$ 
\begin{equation}
\young(\hfil,\hfil)\rightarrow\young(\hfil)\oplus\young(\hfil)\oplus\young(\hfil,\hfil)\oplus\bullet=3\oplus3\oplus\overline{3}\oplus1.
\end{equation} 
which gives the anomaly $A(10)_{\mathfrak{su}(5)}=2A(3)+A(\overline{3})+A(1)=2-1+0=1=A(5)_{\mathfrak{su}(5)}$. 

\subsection{Prove $A(D_1\oplus D_2)=A(D_1)+A(D_2)$ and $A(D_1\otimes D_2)=\dim{(D_1)}A(D_1)+\dim{(D_2)}A(D_1)$}
\begin{equation}
\Tr{\left[\{T_a^{D_1\oplus D_2},T_b^{D_1\oplus D_2}\}T_c^{D_1\oplus D_2}\right]}\equiv A(D_1\oplus D_2)d_{abc}
\end{equation}
But, because $D_1\oplus D_2=\begin{pmatrix}D_1&0\\0&D_2\end{pmatrix}$. Thus the generators of $D_1\oplus D_2$ may be decomposed into independent generators $T_a^{D_1'},T_a^{D_2'}$ with dimension $\dim{(D_1\oplus D_2)}$ which act trivially on their respective invariant subspaces. So $D_1'=\begin{pmatrix}D_1&0\\0&0\end{pmatrix}$, $D_2'=\begin{pmatrix}0&0\\0&D_2\end{pmatrix}$ so that $T_a^{D_1'}\cdot T_b^{D_2'}=0$
\begin{align}
\Tr{\left[\{T_a^{D_1\oplus D_2},T_b^{D_1\oplus D_2}\}T_c^{D_1\oplus D_2}\right]}=&\Tr{\left[\{T_a^{D_1'}+T_a^{D_2'},T_b^{D_1'}+T_b^{D_2'}\}(T_c^{D_1'}+T_c^{D_2'})\right]}\\
=&\Tr{\left[\{T_a^{D_1'},T_b^{D_1'}\}T_c^{D_1'}\right]}+\Tr{\left[\{T_a^{D_12'},T_b^{D_2'}\}T_c^{D_2'}\right]}\\
=& A(D_1)d_{abc}+A(D_2)d_{abc}
\end{align}
because $\Tr{[D_1']}=\Tr{[D_1]}$ etc. So, 
\begin{equation}
A(D_1\oplus D_2)=A(D_1)+A(D_2)
\end{equation}

To prove $A(D_1\otimes D_2)=\dim{(D_1)}A(D_2)+\dim{(D_2)}A(D_1)$ we use the fact that $\Tr{[A\otimes B]}=\Tr{[A]}\Tr{[B]}$.

We can decompose the generators
\begin{equation}
T_a^{D_1\otimes D_2}=T_a^{D_1}\otimes\mathbb{I}^{D_2}+\mathbb{I}^{D_1}\otimes T_a^{D_2}
\end{equation}
So, writing in the shorthand form: $T_a^{D_1}\otimes\mathbb{I}^{D_2}=T_a^{D_1}\mathbb{I}^{D_2}$,
\begin{align}
&\Tr{\left[\{T_a^{D_1\otimes D_2},T_b^{D_1\otimes D_2}\}T_c^{D_1\otimes D_2}\right]}\nonumber\\
&=\Tr{\left(\{T_a^{D_1}\mathbb{I}^{D_2}+\mathbb{I}^{D_1} T_a^{D_2},T_b^{D_1}\mathbb{I}^{D_2}+\mathbb{I}^{D_1} T_b^{D_2}\}(T_c^{D_1}\mathbb{I}^{D_2}+\mathbb{I}^{D_1} T_c^{D_2})\right]}\\
&=\Tr\Big[\left(\{T_a^{D_1}\mathbb{I}^{D_2},T_b^{D_1}\mathbb{I}^{D_2}\}+\{T_a^{D_1}\mathbb{I}^{D_2},\mathbb{I}^{D_1}T_b^{D_2}\}+\{\mathbb{I}^{D_1} T_a^{D_2},T_b^{D_1}\mathbb{I}^{D_2}\}+\{\mathbb{I}^{D_1} T_a^{D_2},\mathbb{I}^{D_1}T_b^{D_2}\}\right)\nonumber\\
&\quad\cdot(T_c^{D_1}\mathbb{I}^{D_2}+\mathbb{I}^{D_1} T_c^{D_2})\Big]\\
&=\Tr{\left[\{T_a^{D_1}\mathbb{I}^{D_2},T_b^{D_1}\mathbb{I}^{D_2}\}T_c^{D_1}\mathbb{I}^{D_2}+\{\mathbb{I}^{D_1} T_a^{D_2},\mathbb{I}^{D_1} T_b^{D_2}\}\mathbb{I}^{D_1} T_c^{D_2}\right]}\\
&=\Tr{[\mathbb{I}^{D_2}]}\Tr{\left[\{T_a^{D_1},T_b^{D_1}\}T_c^{D_1}\right]}+\Tr{[\mathbb{I}^{D_1}]}\Tr{\left[\{T_a^{D_2},T_b^{D_2}\}T_c^{D_2}\right]}\\
&=\dim{(D_2)}A(D_1)d_{abc}+\dim{(D_1)}A(D_2)d_{abc}
\end{align}
where the second from last line follows because terms such as
\begin{align}
\Tr{[\{T_a^{D_1}\mathbb{I}^{D_2},\mathbb{I}^{D_1}T_b^{D_2}\}T^{D_1}_c\mathbb{I}^{D_2}]}=2\Tr{[T_a^{D_1}T^{D_1}_cT_b^{D_2}]}=2\Tr{[T_a^{D_1}T^{D_1}_c]}\Tr{[T_b^{D_2}]}=0
\end{align}
 because of the fact that the generators are traceless $\Tr{[T_a^D]}=0$.
 
 Therefore,
 \begin{equation}
 A(D_1\otimes D_2)=\dim{(D_1)}A(D_2)+\dim{(D_2)}A(D_1).
 \end{equation}
\end{document}