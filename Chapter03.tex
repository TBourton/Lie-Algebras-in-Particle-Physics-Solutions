\documentclass[main.tex]{subfiles}
\begin{document}
\subsection{Use the highest weight decomposition to show $\{j\}\otimes\{s\}\sum_{\oplus l=|s-j|}^{s+j}\{l\}$}
$\{j\}\{s\}$ are two spin irreps over a vector spaces $V_j$ and $V_s$ respectively, then the tensor product $\{j\}\otimes\{s\}$ acts on the vector space $V=V_j\otimes V_s$, but the representation $\{j\}\otimes\{s\}=\{j\}\otimes\mathbb{I}+\mathbb{I}\otimes\{s\}$ is not necessarily, and in general is not, an irrep on $V$.
We label spin $j$ states on $V_j$ as $\left|j,m_j\right>$ and spin $s$ states on $V_s$ as $\left|s,m_s\right>$ with $j,s$ the highest weight states, s.t.
\begin{equation}
J_3^{\{j\}\otimes\{s\}}\left(\left|j,m_j\right>\otimes\left|s,m_s\right>\right)=(m_j+m_s)\left(\left|j,m_j\right>\otimes\left|s,m_s\right>\right).
\end{equation}
and
\begin{align}
&J_3^{\{j\}}\left|j,m_j\right>=m_j\left|j,m_j\right>\\
&J_3^{\{s\}}\left|s,m_s\right>=m_s\left|s,m_s\right>.
\end{align}
The tensor product space has a unique highest weight state
\begin{equation}
\left|j,j\right>\otimes\left|s,s\right>\equiv\left|j+s,j+s\right>.
\end{equation}
The main point is that in subspace of $V$, we denote this subspace by $V_{j+s}\subset V$ where the tensor product representation is an irrep. Then we may write the representation on $V$ as a direct sum of irreps $D_1\oplus D_2\oplus ...$.
\begin{equation}
V=V_1\oplus V_{j+s}
\end{equation}
Now, $V_1\subset V$ can be further decomposed. There are two states of highest weight $j+s-1$, either $\left|j,j-1\right>\otimes\left|s,s\right>$ or $\left|j,j\right>\otimes\left|s,s-1\right>$. One of these hgihest weight states lives in the space $V_1$ while the other lives in another subspace where this tensor product is an irrep similarly this subspace is denoted $V_{j+s-1}$. So,
\begin{equation}
V=V_2\oplus V_{j+s-1}\oplus V_{j+s}
\end{equation}
We may keep decomposing in this fashion, for the highest weight state $\left|j+s,j+s-k\right>$ with $k\leq2s$, from which point the tensor product space reduces to a spin state. We have $k+1$ orthogonal subspaces and 1 of these subspaces the tensor product will be an irrep thus we may construct
\begin{equation}
V=V_{|j-s|}\oplus V_{|j-s|+1}\oplus...\oplus V_{j+s}
\end{equation}
In which each subspace transforms irreducibly under the spin $|j-s|,|j-s|+1,...,j+s$ representation respectively. Thus
\begin{align}
\{j\}\otimes\{s\}&=\{|j-s|\}\oplus\{|j-s|+1\}\oplus...\oplus\{j+s\} \\
&=\sum_{\oplus l=|s-j|}^{s+j}{\{l\}}
\end{align}
%
\subsection{Calculate $\exp{{(\img}}$\underline{$r$}$\cdot$\underline{$\sigma$}$)$}
\begin{align}
\e^{\img\underline{r}.\underline{\sigma}}&=\e^{\img|\underline{r}|\underline{\hat{r}}.\underline{\sigma}}\\
&=\sum_{n=0}^{n=\infty}{\frac{1}{n!}(\img|\underline{r}|)^n(\underline{\hat{r}}.\underline{\sigma})^n}
\end{align}
Pauli matrices satisfy
\begin{equation}
\sigma_i^n = \begin{cases} \sigma_i & \quad \text{if } n \text{ odd}\\ \mathbb{I} & \quad \text{if } n \text{ even}\\ \end{cases},
\end{equation}
thus the sum may be expressed as a sum of even \& odd terms.
\begin{align}
\sum_{n=0}^{n=\infty}{\frac{(\img|\underline{r}|)^n(\underline{\hat{r}}.\underline{\sigma})^n}{n!}}&=\sum_{n=0}^{n=\infty}{\frac{(\img|\underline{r}|)^{2n}(\underline{\hat{r}}.\underline{\sigma})^{2n}}{(2n)!}}+\sum_{n=0}^{n=\infty}{\frac{(\img|\underline{r}|)^{2n+1}(\underline{\hat{r}}.\underline{\sigma})^{2n+1}}{(2n+1)!}}\\
&=\sum_{n=0}^{n=\infty}{\frac{(-1)^{n}|\underline{r}|^{2n}}{(2n)!}}\mathbb{I}+\img\sum_{n=0}^{n=\infty}{\frac{(-1)^n|\underline{r}|^{2n+1}}{(2n+1)!}}(\underline{\hat{r}}.\underline{\sigma})\\
&=\cos{(|\underline{r}|)}\mathbb{I}+\img\sin{(|\underline{r}|)}\underline{\hat{r}}.\underline{\sigma}
\end{align}
\subsection{Show that the spin 1 rep is equivalent to the adjoint rep with $f_{abc}=\epsilon_{abc}$}
The spin 1 rep of $\mathfrak{su}(2)$ is given by the 3 matrices
\begin{equation}
J_1^1=\frac{1}{\sqrt{2}}\begin{pmatrix} 0 & 1 & 0  \\ 1 & 0 & 1 \\ 0 & 1 & 0  \end{pmatrix},\quad
J_2^1=\frac{1}{\sqrt{2}}\begin{pmatrix} 0 & -\img & 0  \\ \img & 0 & -\img \\ 0 & \img & 0  \end{pmatrix},\quad
J_3^1=\frac{1}{\sqrt{2}}\begin{pmatrix} 1 & 0 & 0  \\ 0 & 0 & 0 \\ 0 & 0 & -1  \end{pmatrix}.
\end{equation}
The adjoint representation is generated by the structure constants themselves ${[T_a^{adj}]}_{bc}=-\img f_{abc}=-\img\epsilon_{abc}$. So,
\begin{equation}
T_1^{adj}=-\img\begin{pmatrix} 0 & 0 & 0  \\ 0 & 0 & 1 \\ 0 & -1 & 0  \end{pmatrix},\quad
T_2^{adj}=-\img\begin{pmatrix} 0 & 0 & -1  \\ 0 & 0 & 0 \\ 1 & 0 & 0  \end{pmatrix},\quad
T_3^{adj}=-\img\begin{pmatrix} 0 & 1 & 0  \\ -1 & 0 & 0 \\ 0 & 0 & 0  \end{pmatrix}.
\end{equation}
If the two representations are equivalent then $\exists S \st S:J_a^1\rightarrow T_a^{adj}$. Where $S$ acts "adjointly" on $T_a^{adj}$, i.e. find a matrix $S\st J_a^1=S^{-1}T_a^{adj}S$. Let \begin{equation}
S=\begin{pmatrix}  a & b & c  \\ d & e & f \\ g & h & i  \end{pmatrix}.
\end{equation} 
Then
\begin{align}
SJ^1_1=&\frac{1}{\sqrt{2}}\begin{pmatrix}  b & a+c & b  \\ e & d+f & e \\ h & g+i & h  \end{pmatrix}.\\
T^{adj}_1S=&-\img\begin{pmatrix}  0 & 0 & 0  \\ g & h & i \\ -d & -e & -f  \end{pmatrix}.
\end{align}
Solving the algebra we get 
\begin{equation}
S=\begin{pmatrix}  a & 0 & -a  \\ d & e & d \\ \frac{\img}{\sqrt{2}}e & \img\sqrt{2}d & \frac{\img}{\sqrt{2}e}  \end{pmatrix}.
\end{equation}
Then 
\begin{align}
SJ^2_1=&\frac{1}{\sqrt{2}}\begin{pmatrix}  0 & -\img2a & 0  \\ \img e & 0 & -\img e \\ -\sqrt{2}d & 0 & \sqrt{2}d  \end{pmatrix}.\\
T^{adj}_2S=&-\img\begin{pmatrix}  -\frac{\img}{\sqrt{2}}e & -\img\sqrt{2}d &  -\frac{\img}{\sqrt{2}}e \\ 0 & 0 & 0 \\ a & 0 & -a  \end{pmatrix}.
\end{align}
Which by comparing and setting $a=1$ w.l.o.g. gives us a final form for $S$
\begin{equation}
S=\begin{pmatrix}  1 & 0 & -1  \\ \img & 0 & \img \\0 & -\sqrt{2} & 0 \end{pmatrix}.
\end{equation}
Which one may verify that an $S$ of this form satisfies $J_a^1=S^{-1}T_a^{adj}S\forall a$.
\subsection{Write out the matrix elements of $\sigma_2\otimes\eta_1$.}
The formula we need is given by equation (1.105) in Georgi.
\begin{equation}
[D_{D_1\otimes D_2 (g)}]_{jxky}=\left<j,x\right|D_{D_1\otimes D_2 (g)}\left|k,y\right>\equiv\left<j\right|D_1(g)\left|k\right>\left<x\right|D_2(g)\left|y\right>
\end{equation}
We are given a basis
\begin{align}
\left|1\right>&=\left|i=1\right>\left|x=1\right>\equiv\begin{pmatrix}1\\0\\0\\0\end{pmatrix},\quad
\left|2\right>=\left|i=1\right>\left|x=2\right>\equiv\begin{pmatrix}0\\1\\0\\0\end{pmatrix},\\
\left|3\right>&=\left|i=2\right>\left|x=1\right>\equiv\begin{pmatrix}0\\0\\1\\0\end{pmatrix},\quad
\left|4\right>=\left|i=2\right>\left|x=2\right>\equiv\begin{pmatrix}0\\0\\0\\1\end{pmatrix}.
\end{align}
The non-vanishing matrix elements are
\begin{align}
[\sigma_2\otimes\eta_1]_{14}&=\left<1\right|\sigma_2\left|2\right>\left<1\right|\eta_1\left|2\right>=-\img\\
[\sigma_2\otimes\eta_1]_{23}&=\left<1\right|\sigma_2\left|2\right>\left<2\right|\eta_1\left|1\right>=-\img\\
[\sigma_2\otimes\eta_1]_{32}&=\left<2\right|\sigma_2\left|1\right>\left<1\right|\eta_1\left|2\right>=\img\\
[\sigma_2\otimes\eta_1]_{41}&=\left<2\right|\sigma_2\left|1\right>\left<2\right|\eta_1\left|1\right>=\img.
\end{align}
Which gives the matrix
\begin{equation}
\sigma_2\otimes\eta_1=\begin{pmatrix}0&0&0&-\img\\0&0&-\img&0\\0&\img&0&0\\\img&0&0&0\end{pmatrix}
\end{equation}
Which, in this basis, is completely equivalent to just taking the tensor product of the two matrices in the usual way.
\begin{equation}
\sigma_2\otimes\eta_1=\begin{pmatrix}0&-\img\eta_1\\\img\eta_1&0\end{pmatrix}=\begin{pmatrix}0&0&0&-\img\\0&0&-\img&0\\0&\img&0&0\\\img&0&0&0\end{pmatrix}.
\end{equation}
\subsection{Tensor product notation}
\subsubsection{$[\sigma_a,\sigma_b\eta_c]$}
\begin{align}
[\sigma_a,\sigma_b\eta_c]&=\sigma_a\sigma_b\otimes\eta_c-\sigma_b\sigma_a\otimes\eta_c\\
&=[\sigma_a,\sigma_b]\otimes\eta_c\\
&=2\img\epsilon_{abd}\sigma_d\otimes\eta_c.
\end{align}
\subsubsection{$\Tr{(\sigma_a\{\eta_b,\sigma_c\eta_d\})}$}
\begin{align}
\Tr{[\sigma_a\{\eta_b,\sigma_c\eta_d\}]}&=\Tr{[\sigma_a\sigma_c\otimes\{\eta_b,\eta_d\}]}\\
&=\Tr{[(\delta_{ac}\mathbb{I}+\img\epsilon_{ace}\sigma_e)\otimes2\delta_{bd}\mathbb{I}]}\\
&=8\delta_{ac}\delta_{bd}.
\end{align}
Where we have used $\Tr{[\mathbb{I}\otimes\mathbb{I}]}=4$ and $\Tr{[\sigma_a]}=0$.
\subsubsection{$[\sigma_1\eta_1,\sigma_2\eta_2]$}
\begin{align}
[\sigma_1\eta_1,\sigma_2\eta_2]&=\sigma_1\sigma_2\otimes\eta_1\eta_2-\sigma_2\sigma_1\otimes\eta_2\eta_1\\
&=\img\sigma_3\otimes\img\eta_3-\img\sigma_3\otimes\img\eta_3\\
&=0.
\end{align}
\end{document}