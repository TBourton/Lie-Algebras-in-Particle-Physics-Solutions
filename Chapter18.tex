\documentclass[main.tex]{subfiles}
\begin{document}

\subsection{Check explicitly that mass terms for the $e^-$ and $u,d$ quarks are allowed in the $\mathfrak{su}(2)\times\mathfrak{u}(1)$ Higgs model}
$[R_a,\phi_r^{\dagger}]=\phi_s^{\dagger}[\sigma_a]_{sr}/2$, $[S,\phi_r^{\dagger}]=\phi_r^{\dagger}/2$. Higgs can "produce" a mass via SSB provided the tensor product of a RH particle and a RH anti-particle $\ni$ Higgs rep $(1,2)_{1/2}$ or its conjugate rep. For a RH electron 
\begin{align}
(1,1)_{-1}\otimes(1,2)_{1/2}=(1,2)_{-1/2}
\end{align}
But, as $2=\overline{2}$ for $\mathfrak{su}(2)$, so that $(1,2)_{-1/2}=(1,\overline{2})_{-1/2}=\overline{(1,2)_{1/2}}$ which is the conjugate Higgs rep.

For $u$ quark, using $\overline{3}\otimes3=8\oplus1$
\begin{equation}
(3,1)_{2/3}\otimes(\overline{3},2)_{-1/6}=(8,2)_{1/2}\oplus(1,2)_{1/2}
\end{equation}
which therefore contains the Higgs rep.

For the $d$ quarks
\begin{equation}
(3,1)_{-1/3}\otimes(\overline{3},2)_{-1/6}=(8,2)_{-1/2}\oplus(1,2)_{-1/2}=(8,2)_{-1/2}\oplus\overline{(1,2)_{1/2}}
\end{equation}
Which therefore contains the conjugate Higgs rep.

\subsection{Find the symmetric tensor product $\left((3,1)_{-1/3}\oplus(1,2)_{1/2}\right)$ with itself}

\begin{equation}
\left((3,1)_{-1/3}\oplus(1,2)_{1/2}\right)\oplus_{sym}\left((3,1)_{-1/3}\oplus(1,2)_{1/2}\right)
\end{equation}

In $\mathfrak{su}(3)$
\begin{align}
3\otimes3&=6\oplus\overline{3}\\
&=\yng(2)\oplus\yng(1,1)\\
&=\text{Sym}\oplus\text{AntiSym.}
\end{align}

\begin{align}
3\otimes\overline{3}&=1\oplus8\\
&=\yng(1,1,1)\oplus\yng(2,1)\\
&=\text{AntiSym}\oplus\text{AntiSym.}
\end{align}

In $\mathfrak{su}(2)$
\begin{align}
2\otimes2&=3\oplus1\\
&=\yng(2)\oplus\yng(1,1)\\
&=\text{Sym}\oplus\text{AntiSym.}
\end{align}

So 
\begin{equation}
\left((3,1)_{-1/3}\oplus(1,2)_{1/2}\right)\otimes_{sym}\left((3,1)_{-1/3}\oplus(1,2)_{1/2}\right)=(6,1)_{-2/3}+(3,2)_{1/6}+(1,3)_{1}
\end{equation}

\subsection{Find the symmetric tensor product $\left((3,1)_{2/3}\oplus(1,1)_{-1}\oplus(\overline{3},2)_{-1/6}\right)$ with itself}

\begin{align}
&\left((3,1)_{2/3}\oplus(1,1)_{-1}\oplus(\overline{3},2)_{-1/6}\right)\otimes_{sym}\left((3,1)_{2/3}\oplus(1,1)_{-1}\oplus(\overline{3},2)_{-1/6}\right)\nonumber\\
&=(6,1)_{4/3}\oplus(1,1)_{-2}\oplus(3,1)_{-1/3}\oplus(\overline{3},2)_{-7/6}\oplus(\overline{6},1)_{-1/3}\oplus(3,3)_{-1/3}.
\end{align}

\subsection{Show that if the operator $O=\overline{e}^{\dagger}\epsilon^{abc}u_au_bd_c$ appears in the Hamiltonian it has the correct charge and color properties to allow $P\rightarrow\pi^0e^+$ decay.}
\begin{align}
&\left|P\right>\rightarrow\left|\pi^0e^+\right>\\
&\left|uud\right>\rightarrow\frac{1}{\sqrt{2}}\left(\left|u\overline{u}\right>-\left|d\overline{d}\right>\right)\otimes\left|e^+\right>
\end{align}

$O$ transforms under the $(1,1)_{-1}$ representation. It is therefore colorless transforming trivially under $\mathfrak{su}(3)_c$ and it can therefore couple to the colorless proton, likewise the  $\left|\pi^0e^+\right>$ state is also colorless.
Applying the charge operator $Q$ to the states
\begin{align}
Q\left|P\right>=&+1\left|P\right>\\
Q\left|\pi^0e^+\right>=&+1\left|\pi^0e^+\right>\\
QO=&+1-\frac{2}{3}-\frac{2}{3}+\frac{1}{3}=0.
\end{align}
Thus $O$ is chargeless as required then $O\left|P\right>\rightarrow\left|\pi^0e^+\right>$ is allowed because in $\mathfrak{su}(5)$, $u,d\in$ the same rep. as $e^+$, which is not true in the standard model $\mathfrak{su}(3)\times\mathfrak{su}(2)\times\mathfrak{u}(1)$ theory.


\subsection{How do the $45$ and $50$ of $\mathfrak{su}(5)$ transform under $\mathfrak{su}(3)\otimes\mathfrak{su}(2)\otimes\mathfrak{u}(1)$ subgroup?}
It is easier to work with the conjugate representations and then conjugate again at the end. Under $\mathfrak{su}(5)\rightarrow\mathfrak{su}(3)\times\mathfrak{su}(2)\times\mathfrak{u}(1)$ 
\begin{equation}\label{eq:45decomp}
\begin{split}
\overline{45}=\yng(2,1,1)\rightarrow&(\yng(2,1,1),\bullet)\oplus(\yng(2,1),\yng(1))\oplus(\yng(1,1,1),\yng(1))\oplus(\yng(1,1),\yng(1,1))\\
&\oplus(\yng(1,1),\yng(2))\oplus(\yng(2),\yng(1,1))\oplus(\yng(1),\yng(2,1)).
\end{split}
\end{equation}
Taking the conjugate of \eqref{eq:45decomp} gives
\begin{align}
45=\yng(2,2,1,1)\rightarrow&(\overline{3},1)\oplus(8,2)\oplus(1,2)\oplus(3,1)\oplus(3,3)\oplus(\overline{6},1)\oplus(\overline{3},2).
\end{align}

For the $50$
\begin{equation}
\begin{split}
\overline{50}=\yng(2,2)\rightarrow&(\yng(2,2),\bullet)\oplus(\yng(2,1),\yng(1))\oplus(\yng(1),\yng(2,1))\oplus(\yng(1,1),\yng(1,1))\\
&\oplus(\yng(1,1),\yng(2))\oplus(\bullet,\yng(2,2)).
\end{split}
\end{equation}

Taking the conjugate gives
\begin{align}
50\rightarrow&(6,1)\oplus(8,2)\oplus(\overline{3},1)\oplus(3,1)\oplus(3,3)\oplus(1,1).
\end{align}
\end{document}