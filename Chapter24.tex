\documentclass[main.tex]{subfiles}
\begin{document}
\subsection{Show that the matrices generate a spinor representation of $\mathfrak{so}(10)$. Find an $\mathfrak{su}(2)\times\mathfrak{su}(2)\times\mathfrak{su}(4)$ subgroup}
The matrices are
\begin{equation}
\frac{1}{2}\sigma_i,\quad\frac{1}{2}\tau_i,\quad \frac{1}{2}\eta_i,\quad \frac{1}{2}\sigma_i\rho_1,\quad\frac{1}{2}\tau_i\rho_2,\quad\frac{1}{2}\eta_i\rho_3,\quad\frac{1}{2}\sigma_i\tau_j\rho_3,\quad\frac{1}{2}\tau_i\eta_j\rho_1,\quad\frac{1}{2}\sigma_i\eta_j\rho_2.
\end{equation}
Take the Cartan subalgebra to be 
\begin{align}
H_1=&\frac{1}{2}\sigma_3,\\
H_2=&\frac{1}{2}\tau_3,\\
H_3=&\frac{1}{2}\eta_3,\\
H_4=&\frac{1}{2}\eta_3\rho_3,\\
H_5=&\frac{1}{2}\sigma_3\tau_3\rho_3.
\end{align}
which clearly all have eigenvalues $\lambda=\pm1/2$ and the eigenvectors are given by the unit vectors $[e^k]_m=e^k_m$. Thus, as on page 238 in Georgi, the simple roots are $\alpha^j=e^j-e^{j+1}$ for $j=1...4$ and $\alpha^5=e^4+e^{5}$. So that the fundamental weights, satisfying \eqref{eq:8fundamentalweights} are given by 
\begin{align}
\mu^1=&(1,0,0,0,0)\\
\mu^2=&(1,1,0,0,0)\\
\mu^3=&(1,1,1,0,0)\\
\mu^4=&\frac{1}{2}(1,1,1,1,-1)\\
\mu^5=&\frac{1}{2}(1,1,1,1,1).
\end{align}

For the $\mathfrak{su}(4)\times\mathfrak{su}(2)\times\mathfrak{su}(2)'$ subgroup, the $\mathfrak{su}(2)'$ may be generated by the subset $\eta_i(1+\rho_3)/4=\frac{1}{2}\mathbb{I}\otimes\mathbb{I}\otimes\eta_i\otimes P_1$ with $P_1=\begin{pmatrix}1&0\\0&0\end{pmatrix}$. This has the commutation relations of $\mathfrak{su}(2)$
\begin{equation}
\frac{1}{4}[\eta_iP_1,\eta_jP_1]=\frac{1}{4}[\eta_i,\eta_j]P_1=\frac{\img}{2}\epsilon_{ijk}\eta_kP_1.
\end{equation}
The other $\mathfrak{su}(2)$ factor may be taken to be generated by $\eta_i(1-\rho_3)/4$.

The $\mathfrak{su}(4)$ factor is generated by the $4^2-1=15$ matrices which we can take to be $\frac{1}{2}\sigma_i,\frac{1}{2}\tau_i,\frac{1}{2}\sigma_i\tau_j\rho_3$, these have the commutation relations
\begin{align}
\frac{1}{4}[\sigma_i,\sigma_j]=&\frac{\img}{2}\epsilon_{ijk}\sigma_k\\
\frac{1}{4}[\tau_i,\tau_j]=&\frac{\img}{2}\epsilon_{ijk}\tau_k\\
\frac{1}{4}[\sigma_i,\tau_j]=&0\\
\frac{1}{4}[\sigma_i,\sigma_j\tau_k\rho_3]=&\frac{\img}{2}\epsilon_{ijl}\sigma_l\tau_k\rho_3\\
\frac{1}{4}[\tau_i,\sigma_j\tau_k\rho_3]=&\frac{\img}{2}\epsilon_{ikl}\sigma_j\tau_l\rho_3\\
\frac{1}{4}[\sigma_i\tau_k\rho_3,\sigma_j\tau_l\rho_3]=&\frac{\img}{2}\epsilon_{ijm}\delta_{kl}+\frac{\img}{2}\epsilon_{klm}\delta_{ij}\tau_m
\end{align}
which are the same commutations of those of the $\mathfrak{su}(4)$ found in problem (22.D)

\subsection{What is the dimension of the $\mathfrak{so}(10)$ representation with highest weight $2\mu^5$}
We can use the Weyl dimension (character) formula \cite{Cahn:1985wk}
\begin{equation}
\dim{(R)}=\prod_{\alpha>0}\frac{\sum_ik^i_{\alpha}(\Lambda_i+1)\alpha^i\cdot\alpha^i}{\sum_ik^i_{\alpha}\alpha^i\cdot\alpha^i}
\end{equation}
where $\alpha=\sum_ik_{\alpha}^i\alpha^i$ are positive roots and $\Lambda_i$'s are the Dynkin coefficients of the highest weight states.

Because we are interested in $\mathfrak{so}(10)=D_5$ the $\alpha^i\cdot\alpha^i$ factors cancel because all the roots have the same length.

The simple roots of $\mathfrak{so}(10)$ are given by $\alpha^j=e^j-e^{j+1}$ for $j=1...4$ and $\alpha^5=e^4+e^{5}=(0,0,0,1,1)$. The 20 positive roots are given by $e^j\pm e^k$ for $j<k$.
So we associate with the fundamental weight $2\mu^5$ the vector $(0,0,0,0,2)=\Lambda$.

Denoting the positive roots in terms of the simple roots as $\alpha^1\equiv(1)$,..., $\alpha^5\equiv(5)$, so, for example $\alpha^1+\alpha^2=(1)(2)$. The characters which are not equal to unity, for example $(34):=\frac{1(0+1)+1(0+1)}{2}=1$, are given for the positive roots
\begin{align}
(5):&\qquad  \frac{\sum_ik^i_{\alpha}(\Lambda_i+1)}{\sum_ik^i_{\alpha}}=\frac{1(2+1)}{1}=3\\
(3)(5):&\qquad \frac{1(2+1)+1(0+1)}{2}=2\\
(2)(3)(5):&\qquad \frac{1(2+1)+1(0+1)+1(0+1)}{3}=\frac{5}{3}\\
(3)(4)(5):&\qquad \frac{1(2+1)+1(0+1)+1(0+1)}{3}=\frac{5}{3}\\
(1)(2)(3)(5):&\qquad \frac{3+1+1+1}{4}=\frac{6}{4}\\
(2)(3)(4)(5):&\qquad \frac{3+1+1+1}{4}=\frac{6}{4}\\
(1)(2)(3)(4)(5):&\qquad \frac{3+1+1+1+1}{5}=\frac{7}{5}\\
(2)(3^2)(5):&\qquad \frac{3+1+1+1+1}{5}=\frac{7}{5}\\
(1)(2)(3^2)(4)(5):&\qquad \frac{3+1+1+1+1+1}{6}=\frac{8}{6}\\
(1)(2^2)(3^2)(4)(5):&\qquad \frac{3+1+1+1+1+1+1}{7}=\frac{9}{7}.
\end{align}
Then it is just a case of multiplying all the characters together which gives $\dim{(2\mu^5)}=126$
\end{document}