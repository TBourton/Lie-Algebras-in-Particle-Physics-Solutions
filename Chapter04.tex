\documentclass[main.tex]{subfiles}
\begin{document}
\subsection{Calculate $\left<3/2,-3/2,\alpha\right|O_{-1/2}\left|1,-1,\beta\right>$ given $\left<3/2,-1/2,\alpha\right|O_{1/2}\left|1,-1,\beta\right>=A$}
The operator $O_x$ transforms in the spin $1/2$ representation
\begin{equation}
[J_a,O_x]=\frac{1}{2}O_y[\sigma_a]_{yx}
\end{equation}
With $x,y$ running from $1/2$ to $-1/2$. So $[J_3,O_{1/2}]=\frac{1}{2}O_{1/2}$ and $[J_3,O_{-1/2}]=-\frac{1}{2}O_{-1/2}$. We also have $[J^{\pm},O_{1/2}]=\frac{1}{2\sqrt{2}}(O_{-1/2}\mp O_{-1/2})$ and $[J^{\pm},O_{-1/2}]=\frac{1}{2\sqrt{2}}(O_{1/2}\pm O_{1/2})$.
Thus we know that $O_{1/2}$ has highest weight with $+\frac{1}{2}$ $J_3$ value, thus the highest weight state of the $1\otimes\frac{1}{2}$ tensor product space is 
\begin{equation}
\left|3/2,3/2\right>\equiv O_{1/2}\left|1,1,\beta\right>.
\end{equation}
%Now apply the lowering operators
%\begin{align}
%\left|3/2,-1/2\right>&=\frac{1}{\sqrt{3}}J^-J^-\left|3/2,3/2\right>\\
%&=\frac{1}{\sqrt{3}}J^-J^-O_{1/2}\left|1,1,\beta\right>\\
%&=\frac{1}{\sqrt{3}}J^-([J^-,O_{1/2}]+O_{1/2}J^-)\left|1,1,\beta\right>\\
%&=\frac{1}{\sqrt{3}}J^-(\frac{1}{\sqrt{2}}O_{-1/2}\left|1,1,\beta\right>+O_{1/2}\left|1,0,\be%ta\right>)\\
%&=\frac{2}{\sqrt{6}}O_{-1/2}\left|1,0,\beta\right>+\frac{1}{\sqrt{3}}O_{1/2}\left|1,-1,\beta\%right>.
%\end{align}
%\begin{align}
%\left|3/2,-3/2\right>&=\sqrt{\frac{2}{3}}J^-\left|3/2,-1/2,\beta\right>\\
%&=\sqrt{\frac{2}{3}} J^-(\frac{2}{\sqrt{6}}O_{-1/2}\left|1,0\right>+\frac{1}{\sqrt{3}}O_{1/2}%\left|1,-1\right>)\\
%&=\sqrt{\frac{2}{3}}(\sqrt{\frac{2}{3}}O_{-1/2}\left|1,-1,\beta\right>+\sqrt{\frac{1}{2}}O_{1%/2}\left|1,-1,\beta\right>).
%\end{align}
%Thus,
%\begin{align}
%\left<3/2,-3/2,\alpha\right|O_{-1/2}\left|1,-1,\beta\right>=\frac{\sqrt{6}\sqrt{3}}{\sqrt{2}(%2+\sqrt{3})}\delta_{\alpha\beta}.
%\end{align}
%and 
%\begin{align}
%\left<3/2,-1/2,\alpha\right|O_{-1/2}\left|1,-1,\beta\right>=-\frac{\sqrt{3}}{2}A.
%\end{align}
We have
\begin{align}
A=&\left<3/2,-1/2,\alpha\right|O_{1/2}\left|1,-1,\beta\right>\\
=&\sqrt{\frac{2}{3}}\left<3/2,-3/2,\alpha\right|J^-O_{1/2}\left|1,-1,\beta\right>\\
=&\sqrt{\frac{2}{3}}\left<3/2,-3/2,\alpha\right|[J^-,O_{1/2}]\left|1,-1,\beta\right>\\
=&\sqrt{\frac{2}{3}}\left<3/2,-3/2,\alpha\right|\frac{1}{\sqrt{2}}O_{-1/2}\left|1,-1,\beta\right>.
\end{align}
Therefore,
\begin{equation}
\left<3/2,-3/2,\alpha\right|O_{-1/2}\left|1,-1,\beta\right>=\sqrt{3}A.
\end{equation}
\subsection{Construct the components $O_m$.}
We have $[L^+,(r_{+1})^2]=0$ and is therefore the $O_{+2}$ component of a spin 2 operator. The other components are given by applying the lowering operators. We define $O_{j,j}=(r_{+1})^j$ so $(r_{+1})^2\equiv O_{2,2}$ and $[L^-,O_{j,m}]=\sqrt{(j+m)(j-m+1)/2}O_{j,m-1}$. Therefore,
\begin{align}
&O_{2,2}=(r_{+1})^2\\
&O_{2,1}=\frac{1}{\sqrt{2}}[L^-,O_{2,2}]=\frac{1}{\sqrt{2}}([L^-,r_{+1}]r_{+1}+r_{+1}[L^-,r_{+1}])=\frac{1}{\sqrt{2}}(r_{0}r_{+1}+r_{+1}r_{0})\\
&O_{2,0}=\frac{1}{\sqrt{3}}[L^-,O_{2,1}]=\frac{1}{\sqrt{6}}[L^-,(r_{0}r_{+1}+r_{+1}r_{0})]=\frac{1}{\sqrt{6}}(r_{-1}r_{+1}+2(r_0)^2+r_{+1}r_{-1})\\
&O_{2,-1}=\frac{1}{\sqrt{3}}[L^-,O_{2,0}]=\frac{1}{\sqrt{18}}[L^-,(r_{-1}r_{+1}+2(r_0)^2+r_{+1}r_{-1})]=\frac{1}{\sqrt{2}}(r_{-1}r_{0}+r_{0}r_{-1})\\
&O_{2,-2}=\frac{1}{\sqrt{2}}[L^-,O_{2,-1}]=\frac{1}{2}[L^-,(r_{-1}r_{0}+r_{0}r_{-1})]=(r_{-1})^2.
\end{align}

Letting $r_1=\sin{\theta}\cos{\phi}$, $r_2=\sin{\theta}\sin{\phi}$, $r_3=\cos{\theta}$ so that $r_0=r_3=\cos{\theta}$ and $r_{\pm1}=\mp(r_1\pm\img r_2)\sqrt{2}=\mp\frac{\sin{\theta}}{\sqrt{2}}(\cos{\phi}\pm\img\sin{\phi})=\mp\frac{\sin{\theta}}{\sqrt{2}}\e^{\pm\img\phi}$.

Then, looking up the spherical harmonics in a book \cite{foot} we see that
\begin{equation}
r_{1,m}=\sqrt{\frac{4\pi}{3}}Y_{1,m}(\theta,\phi)
\end{equation}
and also
\begin{equation}
O_{2,m}=\sqrt{\frac{8\pi}{15}}Y_{2,m}(\theta,\phi).
\end{equation}

\subsection{Calculate $\e^{\img\alpha_aX_a^1}$}
\begin{equation}
X^1_1=\frac{1}{\sqrt{2}}\begin{pmatrix}0&1&0\\1&0&1\\0&1&0\end{pmatrix},\quad X^2_1=\frac{1}{\sqrt{2}}\begin{pmatrix}0&-\img&0\\\img&0&-\img\\0&\img&0\end{pmatrix},\quad X^3_1=\frac{1}{\sqrt{2}}\begin{pmatrix}1&0&0\\0&0&0\\0&0&-1\end{pmatrix}.
\end{equation}
We write $\alpha_aX_a^1=\alpha\hat\alpha_aX_a^1$ with $\alpha=\sqrt{\alpha_a\alpha_a}$ and $\hat\alpha_a\hat\alpha_a=1$. We are also told that $(\hat\alpha_aX_a^1)^2$ is a projection operator, i.e. $(\hat\alpha_aX_a^1)^{2n}=(\hat\alpha_aX_a^1)^2\Forall n\in\mathbb{Z}$, and therefore $(\hat\alpha_aX_a^1)^{2n+1}=\hat\alpha_aX_a^1$. Thus, we may write the expansion
\begin{align}
\e^{\img\alpha_aX_a^1}&=\sum_{n=0}^{n=\infty}{\frac{(\img\alpha_aX_a^1)^n}{n!}}\\
&=\sum_{n=0}^{n=\infty}{\frac{(\img\alpha_aX_a^1)^{2n}}{(2n)!}}+\sum_{n=0}^{n=\infty}{\frac{(\img\alpha_aX_a^1)^{2n+1}}{(2n+1)!}}\\
&=\sum_{n=0}^{n=\infty}{\frac{(-1)^n(\alpha)^{2n}}{(2n)!}(\hat\alpha_aX_a^1)^2}+\img\sum_{n=0}^{n=\infty}{\frac{(-1)^n(\alpha)^{2n+1}}{(2n+1)!}\hat\alpha_aX_a^1}\\
&=(\hat\alpha_aX_a^1)^2\cos{\alpha}+\img(\hat\alpha_aX_a^1)\sin{\alpha}.
\end{align}
\end{document}