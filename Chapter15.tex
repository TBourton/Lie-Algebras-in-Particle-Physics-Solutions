\documentclass[main.tex]{subfiles}
\begin{document}
\subsection{Find the $\mathfrak{su}(6)$ wave functions for all spin $1/2$ Baryons}
\begin{align}
\left|\Lambda,1/2\right>=\frac{\sqrt{3}}{6}\big[& (\left|uds\right>-\left|dus\right>)(\left|+-+\right>-\left|-++\right>)\nonumber\\
&+(\left|sud\right>-\left|sdu\right>)(\left|++-\right>-\left|+-+\right>)\nonumber\\
&+(\left|dsu\right>-\left|usd\right>)(\left|-++\right>-\left|++-\right>)\big].
\end{align}

\begin{align}
\left|P,1/2\right>=\frac{\sqrt{2}}{6}\big[& \left|uud\right>(2\left|++-\right>-\left|+-+\right>-\left|-++\right>)\nonumber\\
&+\left|udu\right>(2\left|+-+\right>-\left|-++\right>-\left|++-\right>)\nonumber\\
&+\left|duu\right>(2\left|-++\right>-\left|++-\right>-\left|+-+\right>)\big].
\end{align}

\begin{align}
\left|N,1/2\right>=\frac{\sqrt{2}}{6}\big[& \left|udd\right>(2\left|-++\right>-\left|+-+\right>-\left|++-\right>)\nonumber\\
&+\left|dud\right>(2\left|+-+\right>-\left|++-\right>-\left|-++\right>)\nonumber\\
&+\left|ddu\right>(2\left|++-\right>-\left|-++\right>-\left|+-+\right>)\big].
\end{align}

\begin{align}
\left|\Sigma^+,1/2\right>=\frac{\sqrt{2}}{6}\big[& \left|uus\right>(2\left|++-\right>-\left|+-+\right>-\left|++-\right>)\nonumber\\
&+\left|usu\right>(2\left|+-+\right>-\left|++-\right>-\left|-++\right>)\nonumber\\
&+\left|suu\right>(2\left|-++\right>-\left|++-\right>-\left|+-+\right>)\big].
\end{align}

The total wavefunction must be symmetric under exchange of flavour-spin. Whereas $\Lambda$ is an isospin singlet state and is anti-symmetric in both flavour and spin, $\Sigma^0$ is the orthogonal state which is symmetric in both flavour and spin. So, 

\begin{align}
\left|\Sigma^0,1/2\right>=\frac{\sqrt{3}}{6}\big[& (\left|uds\right>+\left|dus\right>)(\left|+-+\right>+\left|-++\right>)\nonumber\\
&+(\left|sud\right>+\left|sdu\right>)(\left|++-\right>+\left|+-+\right>)\nonumber\\
&+(\left|dsu\right>+\left|usd\right>)(\left|-++\right>+\left|++-\right>)\big].
\end{align}

\begin{align}
\left|\Sigma^-,1/2\right>=\frac{\sqrt{2}}{6}\big[& \left|dds\right>(2\left|++-\right>-\left|+-+\right>-\left|-++\right>)\nonumber\\
&+\left|dsd\right>(2\left|+-+\right>-\left|++-\right>-\left|-++\right>)\nonumber\\
&+\left|sdd\right>(2\left|-++\right>-\left|++-\right>-\left|+-+\right>)\big].
\end{align}

\begin{align}
\left|\Xi^0,1/2\right>=\frac{\sqrt{2}}{6}\big[& (\left|ssu\right>(2\left|++-\right>-\left|+-+\right>-\left|-++\right>)\nonumber\\
&+\left|sus\right>(2\left|+-+\right>-\left|++-\right>-\left|-++\right>)\nonumber\\
&+\left|uss\right>(2\left|-++\right>-\left|++-\right>-\left|+-+\right>)\big].
\end{align}

\begin{align}
\left|\Xi^-,1/2\right>=\frac{\sqrt{2}}{6}\big[& \left|ssd\right>(2\left|++-\right>-\left|+-+\right>-\left|-++\right>)\nonumber\\
&+\left|sds\right>(2\left|+-+\right>-\left|++-\right>-\left|-++\right>)\nonumber\\
&+\left|dss\right>(2\left|-++\right>-\left|++-\right>-\left|+-+\right>)\big].
\end{align}


\subsection{Calculate the magnetic moments for the spin $1/2$ Baryons}
\subsubsection{In the $\mathfrak{su}(6)$ limit calculate the ratios to $\mu_P\propto1$}
The magnetic moments are given by
\begin{equation}
\mu_{56}\propto\left<56|Q\sigma_3|56\right>
\end{equation}

The $N$ \& $P$ magnetic moments are given by equations (15.21) \& (15.20) respectively $\mu_P\propto\left<N,1/2|Q\sigma_3|N,1/2\right>=\frac{-2}{3}$, \& $\mu_P\propto\left<P,1/2|Q\sigma_3|P,1/2\right>=1$.
\begin{align}
\mu_{\Lambda}\propto&\left<\Lambda,1/2|Q\sigma_3|\Lambda,1/2\right>\nonumber\\
=&\left<\Lambda,1/2\right|\frac{\sqrt{3}}{18}\Big[(2\left|uds\right>+\left|dus\right>)(\left|+-+\right>-\left|-++\right>)\nonumber\\
&+(-\left|uds\right>-2\left|dus\right>)(-\left|+-+\right>-\left|-++\right>)\nonumber\\
&+(-\left|uds\right>+\left|dus\right>)(\left|+-+\right>-\left|-++\right>)+\text{perms.}\Big]\\
=&-\frac{1}{3}.
\end{align}

\begin{align}
\mu_{\Sigma^+}\propto&\left<\Sigma^+,1/2|Q\sigma_3|\Sigma^+,1/2\right>\nonumber\\
=&\left<\Sigma^+,1/2\right|\frac{\sqrt{2}}{18}\Big[ 2\left|uus\right>(2\left|++-\right>-\left|+-+\right>+\left|++-\right>)\nonumber\\
&+2\left|uus\right>(2\left|++-\right>-\left|+-+\right>-\left|++-\right>)\nonumber\\
&-\left|uus\right>(-2\left|++-\right>-\left|+-+\right>-\left|++-\right>)+\text{perms.}\Big]\\
=&1.
\end{align}

The rest of the magnetic moments are
\begin{align}
&\mu_{\Sigma^0}\propto\left<\Sigma^0,1/2|Q\sigma_3|\Sigma^0,1/2\right>=\frac{1}{3}\\
&\mu_{\Sigma^-}\propto\left<\Sigma^-,1/2|Q\sigma_3|\Sigma^-,1/2\right>=-\frac{1}{3}\\
&\mu_{\Xi^0}\propto\left<\Xi^0,1/2|Q\sigma_3|\Xi^0,1/2\right>=-\frac{2}{3}\\
&\mu_{\Xi^-}\propto\left<\Xi^-,1/2|Q\sigma_3|\Xi^-,1/2\right>=-\frac{1}{3}.
\end{align}
The ratios to $\mu_P$ are:
\begin{align}
&\frac{\mu_{N}}{\mu_{P}}=-\frac{2}{3}\\
&\frac{\mu_{\Lambda}}{\mu_{P}}=-\frac{1}{3}\\
&\frac{\mu_{\Sigma^+}}{\mu_{P}}=1\\
&\frac{\mu_{\Sigma^0}}{\mu_{P}}=\frac{1}{3}\\
&\frac{\mu_{\Sigma^-}}{\mu_{P}}=-\frac{1}{3}\\
&\frac{\mu_{\Xi^0}}{\mu_{P}}=-\frac{2}{3}\\
&\frac{\mu_{\Xi^-}}{\mu_{P}}=-\frac{1}{3}.
\end{align}
Which matches the results found in Problem (11.C).
\subsubsection{Put in $\mathfrak{su}(3)$ symmetry breaking by including $m_s\neq m_{u,d}$.}
Assume $m\simeq\frac{1}{3}m_p\simeq\frac{1}{3}m_d\simeq310$MeV and $m_s\simeq490$MeV. Using $\frac{\mu_x}{m_x}=\left<x|\frac{1}{m_{quark}}Q\sigma_3|x\right>/\left<x|\sigma_3|x\right>$. For example
\begin{align}
\frac{\mu_P}{m_P}=&\frac{\left<P|\sum_{quarks}\frac{Q\sigma_3}{m_{quark}}|P\right>}{\left<P|\sigma_3|P\right>}\\
=&\left<P\right|\frac{\sqrt{2}}{6}\big[ \frac{2}{3m_u}\left|uud\right>(2\left|++-\right>-\left|+-+\right>+\left|-++\right>)\nonumber\\
&-\frac{1}{3m_d}\left|udu\right>(-2\left|+-+\right>-\left|-++\right>-\left|++-\right>)\nonumber\\
&+\frac{2}{3m_u}\left|duu\right>(2\left|-++\right>+\left|++-\right>-\left|+-+\right>)\big]\\
=&\frac{1}{m}\\
\implies&\mu_P=\frac{3m}{m}=3.
\end{align}
Repeating, in the same fashion, for the others we find
\begin{align}
&\mu_N=m_N\frac{-2}{3m}=-2\\
&\mu_{\Lambda}=\frac{-m_{\Lambda}}{3m_s}\simeq\frac{-m_s-2m}{3m_s}\simeq-0.75\\
&\mu_{\Sigma^+}=-\frac{m_{\Sigma}}{6}(\frac{16}{3m}+\frac{2}{3m_s})\simeq(\frac{8}{9m}+\frac{1}{9m_s})(2m+m_s)\simeq3.4\\
&\mu_{\Sigma^0}=\frac{m_{\Sigma^0}}{3m_s}\simeq\frac{m_s+2m}{3m_s}\simeq0.75\\
&\mu_{\Sigma^-}=\frac{m_{\Sigma^-}}{6}(\frac{-8}{3m}+\frac{2}{3m_s})\simeq(\frac{1}{m_s}-\frac{4}{9m})(2m+m_s)\simeq-1.33\\
&\mu_{\Xi^0}=\frac{m_{\Xi^0}}{6}(-\frac{8}{3m_s}-\frac{4}{3m})\simeq-(\frac{4}{9m_s}+\frac{2}{9m})(2m_s+m)\simeq-2.11\\
&\mu_{\Xi^-}=\frac{m_{\Xi^-}}{6}(\frac{-8}{3m_s}+\frac{2}{3m})\simeq(\frac{1}{9m}-\frac{4}{9m_s})(2m_s+m)\simeq-0.71.
\end{align}

\subsection{Show that $\left|\Lambda,1/2\right>$ is an isospin singlet}
To show this we must first show that the state has isospin $0$ and is also annihilated by the raising and lowering operators of the isospin algebra. The isospin operator on $\mathfrak{su}(6)$ is $I_3\otimes\mathbb{I}=\frac{1}{2}\lambda_3\otimes\mathbb{I}$.
\begin{equation}
I_3\left|u\right>=\frac{1}{2}\begin{pmatrix}1&0&0\\0&-1&0\\0&0&0\end{pmatrix}\left|\begin{pmatrix}1\\0\\0\end{pmatrix}\right>=\frac{1}{2}\left|u\right>
\end{equation}
Similarly, $I_3\left|d\right>=-\frac{1}{2}\left|d\right>$, $I_3\left|s\right>=0\left|s\right>=0$. So,
\begin{align}
I_3\otimes\mathbb{I}\left|\Lambda,1/2\right>=&\frac{\sqrt{3}}{6}\Big((\frac{1}{2}-\frac{1}{2}+0)\left|uds\right>-(-\frac{1}{2}+\frac{1}{2}+0)\left|dus\right>\Big)\\&\otimes(\left|+-+\right>-\left|-++\right>) + \text{perms}\\
=&0\otimes(\left|+-+\right>-\left|-++\right>) + 0\otimes\text{perms}.
\end{align}
Then, by applying the ladder operators $E_{\pm1,0}=\frac{1}{2}(\lambda_1\pm\img\lambda_2)$, $E_{\pm1/2,\pm\sqrt{3}/2}=\frac{1}{2}(\lambda_4\pm\img\lambda_5)$, $E_{\mp1/2,\pm\sqrt{3}/2}=\frac{1}{2}(\lambda_6\pm\img\lambda_7)$
Applying these operators to the flavour states, the nonvanishing possibilities are
\begin{align}
&E_{-1,0}\left|u\right>=\left|d\right>\\
&E_{+1,0}\left|d\right>=\left|u\right>\\
&E_{-1/2,-\sqrt{3}/2}\left|u\right>=\left|s\right>\\
&E_{+1/2,+\sqrt{3}/2}\left|s\right>=\left|u\right>\\
&E_{-1/2,+\sqrt{3}/2}\left|d\right>=\left|s\right>\\
&E_{+1/2,-\sqrt{3}/2}\left|s\right>=\left|d\right>.
\end{align}
So,
\begin{align}
E_{+1,0}\otimes\mathbb{I}\left|\Lambda,1/2\right>=&\frac{\sqrt{3}}{6}\Big(\left|0ds\right>-\left|uus\right>+\left|uus\right>-\left|d0s\right>+\left|ud0\right>-\left|du0\right>\Big)\\&\otimes(\left|+-+\right>-\left|-++\right>) + \text{perms}\\
=&0\otimes(\left|+-+\right>-\left|-++\right>) + 0\otimes\text{perms}.
\end{align}
Similarly all other raising/lowering operators annihilate the flavour part of the state as the raising/lowering ops will always annihilate 4 of the states and the two remaining states will cancel because of the antisymmetry (compare this with the flavour symmetrical $\Sigma^0$ where the states do not cancel but add.) 
To see they cancel, for example look at
\begin{align}
E_{\mp1/2,\pm\sqrt{3}/2}\left|\Lambda,1/2\right>=&\frac{\sqrt{3}}{6}\Big(\left|0ds\right>-\left|0us\right>+\left|u0s\right>-\left|d0s\right>+\left|udd\right>-\left|dud\right>\Big)\\&\otimes(\left|+-+\right>-\left|-++\right>) + \text{perms}\\
=&0
\end{align}
because the state is symmetrical under the interchange of a pair of flavours $\left|udd\right>=\left|dud\right>$.
\end{document}