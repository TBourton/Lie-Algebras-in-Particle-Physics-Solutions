\documentclass[main.tex]{subfiles}
\begin{document}
\subsection{Calculate $f_{147}$ \& $f_{458}$ in $\mathfrak{su}(3)$}
$T_a=\frac{\lambda_a}{2}$ and $[T_a,T_b]=\img F_{abc}T_c$.
\begin{equation}
[T_1,T_4]=\frac{1}{4}\begin{pmatrix}0&0&0\\0&0&1\\0&-1&0\end{pmatrix}=\img f_{14c}T_c
\end{equation}
Inspecting the other generators we see that $f_{14c}=0$ for $c\neq6,7$, so
\begin{align}
[T_1,T_4]&=\img f_{146}T_6 + \img f_{147}T_7\\
&=\frac{1}{2}\img f_{146}\begin{pmatrix}0&0&0\\0&0&1\\0&1&0\end{pmatrix}+\frac{1}{2}\img f_{147}\begin{pmatrix}0&0&0\\0&0&-\img\\0&\img&0\end{pmatrix}.
\end{align}
This implies that $f_{146}=0$ and $f_{147}=\frac{1}{2}$ because the structure constants are real. Because in some basis $\{e_i\}$ with the basis commutator $[e_i,e_j]=c^k_{ij}e_k$, and since a representation $D$ maps a Lie algebra to a set of linear operators $D:\mathcal{L}\rightarrow\{\text{Linear ops. on V}\}$ with Lie bracket on the generators $X_a$ 
\begin{align}
[X_a,X_b]&=[D(e_a),D(x_b)]\\
&=D([e_a,e_b])\\
&=D(c^k_{ab}e_k)\\
&=c^k_{ab}X_k\\
&\implies f_{abc}\text{'s real}.
\end{align}

\begin{equation}
[T_4,T_5]=\frac{\img}{2}\begin{pmatrix}1&0&0\\0&0&0\\0&0&-1\end{pmatrix}=\img f_{45c}T_c.
\end{equation}
So $f_{45c}=0$ for $c\neq3,8$. Therefore
\begin{align}
[T_4,T_5]&=\img f_{453}T_3 + \img f_{458}T_8\\
&=\frac{1}{2}\img f_{453}\begin{pmatrix}1&0&0\\0&-1&0\\0&0&0\end{pmatrix}+\frac{1}{2\sqrt{3}}\img f_{458}\begin{pmatrix}1&0&0\\0&1&0\\0&0&-2\end{pmatrix}.
\end{align}
Therefore $f_{453}=\frac{1}{\sqrt{3}}f_{458}=\frac{1}{2}$. Thus $f_{458}=\frac{\sqrt{3}}{2}$

\subsection{Show $T_1, T_2, T_3$ generate an $\mathfrak{su}(2)$ subalgebra of $\mathfrak{su}(3)$. How does the representation generated by the Gell-Mann matrices transform under the algebra?}
To show this we must show that $[T_a,T_b]=\img \epsilon_{abc}T_c$, $a,b,c=1..3$ with $T_a=\frac{1}{2}\begin{pmatrix}\sigma_a&0\\0&0\end{pmatrix}$
\begin{align}
[T_a,T_b]&=\frac{1}{2}\begin{pmatrix}[\sigma_a,\sigma_b]&0\\0&0\end{pmatrix}\\
&=\img \epsilon_{abc}T_c
\end{align}
Thus $T_1, T_2, T_3$ does indeed generate an $\mathfrak{su}(2)$ subalgebra of $\mathfrak{su}(3)$. Next we calculate the weights of the subalgebra. Take the Cartan subalgebra to be $T_3$ which has eigenvalues $\Lambda=0,\pm\frac{1}{2}$, then the associated weights are
\begin{align}
&\Lambda_{\frac{1}{2}}=\frac{1}{2} \quad V_{\frac{1}{2}}= \begin{pmatrix}1\\0\\0\end{pmatrix}\\
&\Lambda_0=0 \quad V_0= \begin{pmatrix}0\\0\\1\end{pmatrix}\\
&\Lambda_{-\frac{1}{2}}=-\frac{1}{2} \quad V_{-\frac{1}{2}}= \begin{pmatrix}0\\1\\0\end{pmatrix}\\
\end{align}
To determine whether the $\mathfrak{su}(3)$ representation is an irrep under the $\mathfrak{su}(2)$ subalgebra we look for invariant subspaces by applying the raising and lowering operators $T^{\pm}\equiv(T_1\pm \img T_2)$.
\begin{align}
T^+V_{\frac{1}{2}}&=0 \quad \implies V_{\frac{1}{2}} \quad\text{highest weight state}\\
T^-V_{\frac{1}{2}}&=V_{-\frac{1}{2}}\\
T^+V_{-\frac{1}{2}}&=V_{\frac{1}{2}}\\
T^-V_{-\frac{1}{2}}&=0\\
T^{\pm}V_0&=0.
\end{align}
Thus the $\mathfrak{su}(2)$ subalgebra is not irreducible on the $\mathfrak{su}(3)$ representation generated by the Gell-Mann matrices and the subalgebra decomposes into $3=2\oplus1$.

\subsection{Show $\lambda_2, \lambda_5, \lambda_7$ generate an $\mathfrak{su}(2)$ subalgebra of $\mathfrak{su}(3)$. How does the representation generated by the Gell-Mann matrices transform under the algebra?}
We can again easily show that $[\lambda_a,\lambda_b]=\img \epsilon_{abc}\lambda_c$, $a,b,c=2,5,7$ and thus generate an $\mathfrak{su}(2)$ subalgebra. We pick $\lambda_2$ as the Cartan subalgebra and $\lambda^{\pm}\equiv\frac{1}{\sqrt{2}}(\lambda_5\pm\img\lambda_7)$ so that $[\lambda_2,\lambda^{\pm}]=\pm\lambda^{\pm}$. The eigenvalues are given by $\Lambda=0,\pm1$. The associated weights are
\begin{align}
&\Lambda_1=1 \quad V_1= \frac{1}{\sqrt{2}}\begin{pmatrix}-\img\\1\\0\end{pmatrix}\\
&\Lambda_0=0 \quad V_0= \begin{pmatrix}0\\0\\1\end{pmatrix}\\
&\Lambda_{-1}=-1 \quad V_{-1}= \frac{1}{\sqrt{2}}\begin{pmatrix}-\img\\-1\\0\end{pmatrix}.
\end{align}
Applying the raising and lowering operators
\begin{align}
\lambda^+V_{1}&=0 \quad \implies V_{1} \quad\text{highest weight state}\\
\lambda^-V_{1}&=V_{0}\\
\lambda^+V_{0}&=V_{1}\\
\lambda^-V_{0}&=V_{-1}\\
\lambda^+V_{-1}&=V_{0}\\
\lambda^-V_{-1}&=0.
\end{align}
Thus there are no invariant subspaces $\implies$ therefore the $\mathfrak{su}(2)$ subalgebra generated by $\lambda_2, \lambda_5, \lambda_7$ acts irreducibly on the Gell-Mann representation.
\end{document}